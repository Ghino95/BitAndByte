%\documentclass[numbers=noenddot, 12pt, a4paper, oneside]{scrbook}%
\documentclass[12pt, a4paper]{report}
\usepackage{blindtext}
\usepackage{fullpage}
\usepackage[utf8]{inputenc}
\usepackage{float}
\usepackage{hyperref}
\usepackage{hyperref}
\usepackage{tabularx}
\usepackage{graphicx}
\def\Plus{\texttt{+}}
\usepackage{listings}
\usepackage{xcolor}

\begin{document}

\begin{titlepage}
	\centering
	\vspace{1cm}
	\vspace{1cm}

	{\scshape\Large Design Document\par}
	\vspace{0.1cm}
	\begin{figure}[H]
		\centering
		\includegraphics[width=0.5\textwidth]{images/Logo}
	\end{figure}
	\vspace{1cm}
	\vspace{3cm}
	{\Large\itshape di\par}
	{\Large\itshape Gianluigi Oliva, Francesco Musio, Lucia Ferrari\par}
	{\Large\itshape Filippo Ghinelli, Leonardo Febbo\par}
	\vspace{1.5cm}
	\vfill
	


	\vfill

	% Bottom of the page
	{\large \today\par}
\end{titlepage}

\newpage
\tableofcontents
\newpage

\chapter{Overview and General Idea}
\section*{Introduction}
A terrible malware infected Tommy’s computer and he cannot use it anymore. When everything seemed to be lost, here it comes: a help from a little hero…a Bit. Our hero will have to travel to the heart of the pc: the kernel, fixing all the bugs caused by the malware and destroy it. Start with the high-level applications, such as a browser, and go on to the operative system to solve various problems. Free the imprisoned programs and ask them to help you during your adventure.\\

This is a 2D puzzle game with cooperative and platforming elements thought for the PC. The stylistic approach of this game is cartoony. The entire game is subdivided in several macro levels with a certain number of levels for each one.

\section*{Description}
The game is divided into a series of levels with fixed camera in which the player will have to solve puzzles to reach the exit. The levels are organized according to an inverse pyramid structure: at the top of the pyramid there are levels related to the application part of a computer, while going down there are ones related to the operative system and, in the end, the ones related to the kernel.

In addition to the protagonist, the player will be able to control other programs released during the game and exploit their various skills. In the levels there are also enemies and sensors that will immediately trigger an alarm in case of danger and it is necessary not to be spotted. The player can only control one character at a time, so it is better to change control when all the characters are safe. Therefore, in this game, elements of platforming, puzzles and co-op elements are combined.

To complete each level it is necessary that all the characters arrive at the exit door while they are in possession of the key present in the level to open it.

\section*{Audience and Marketing}
The game is thought to be played from people of all ages. In fact, the single puzzles have an overall complexity in order to be entertaining for both casual or hardcore gamers. As said before, the graphic style is cartoony and can be appealing for everyone.\\
The principal competitors of this game are games like “Thomas Was Alone” where there is a strong cooperation factor between the characters, or “Fez” for his puzzle component.

\section*{Genre}
Platformer, Stealth, Puzzle
\section*{Platform(s)}
PC, MacOS
\section*{Number of Player}
Single and Multi Player

\chapter{Principal Game Mechanics and Gameplay}
The principal mechanic of this game is the cooperation between the characters, the different level elements and enemies. In particular using the characters’ skills to create combinations and different interactions within the level.

\section*{Basic skills}
These are the basic skills and mechanics for all the characters that the player can control:
\begin{itemize}
	\item \textbf{Walk}: The basic movement action of the characters. It should give a sense of mass and gravity, without slowing down movement.
	\item \textbf{Jump}: The character should be able to jump but it can be possible to change direction while in air, and gravity should be treated like in the real world. Like for the action of walking, a sense of mass and weight should be present. 
	I\item \textbf{Interacting} with levers, buttons and boxes: The character should be able to interact with predefined machineries and levers with the usage of a single button.
\end{itemize}

The failure in a level is due to the detection or death of one the characters by enemies or sensors. In particular this can happen by:
\begin{itemize}
	\item being detected from a "Resident",
	\item going through a laser sensor,
	\item being hitted from a melee attack,
	\item falling in traps,
	\item ...
\end{itemize}



\section*{Character’s skills}
Each character has a particular skill which can be used in different ways to solve a puzzle:
\begin{itemize}
	\item \textbf{Bitty}: the protagonist and the starting character of the game, he can throw a bit collected around a level (only 1 bit each time). The bits can be used to distract enemies (and to stun the little ones) and to push certain buttons.
	\item \textbf{Shieldy}: a program who can stun enemies from behind. This ability can be turned on or off and covers a circular area around the character.
	\item \textbf{Zippy}:  a program who can shrink allies and enemies blocked by Shieldy, but he can also turn them back to their original size. He can shrink himself horizontally or vertically, gaining respectively height or width.
	\item \textbf{Gimpy}: a program who can camouflage himself to avoid being seen from enemies but he is still tangible. He can camouflage objects or allies too.
\end{itemize}

\section*{Enemys' skills}
There are different enemies in the game, each one representing the different kind of computer virus. Each enemy has a particular feature and behaviour:
\begin{itemize}
	\item \textbf{Resident}: The Resident is a sentinel with a very wide field of vision. If he identifies a danger he launches an alarm that leads to the game over. This enemy is able to move along a predefined path and he is controlled by an AI algorithm.
	\item \textbf{Trojan}: The Trojan is a guard that moves back and forth protecting a specific area in a level. If he spots an enemy, the Trojan will try to attack him with a melee attack by heading towards him at great speed.
	\item \textbf{Spyware}: The Spyware is an enemy hanging from the ceiling that moves from top to bottom. This enemy is not particularly dangerous, but he attacks from above with an incredible speed and, if touched, it leads to the death of the character.
	\item \textbf{Worm}: The Worm is a very large and heavy enemy, almost impossible to move or neutralize. He blocks passages or certain elements of the level as, for example, buttons.
	\item \textbf{Infector}: The Infector is a virus that propagates simply by touching the player or his allies. Once touched the player has little time to cure the infected before turning into a virus. The only one able to cure from this infection is Shieldy, who is also immune to this infection.
	\item \textbf{Macro}: Macro viruses are enemies capable of disguising themselves in common objects, like buttons. However, they are impatient and can be recognized because they occasionally move.
\end{itemize}




\section*{Camera}
Since the game is based on puzzle resolution, the player must always have a complete view of the map, so a static camera is used which always shows the entire level and it's positioned at a fixed distance.\\
Each time the player finishes a level the camera will show the next level, after a small cutscene. The camera will not be static only during the tutorial phase, which consists of a sliding level in which the player will experience all the basic mechanics.

\section*{Hazards}
In the levels of the game, the principal dangers which can be found are:
\begin{itemize}
	\item \textbf{Laser}: Lasers kill anyone who touches them. Some lasers are fixed, while others can rotate either by default or by interacting with a terminal, but they can't be turned off. These are generated by an emitter positioned around the level. These lasers have no effect on anything other than a character or an enemy.
	\item \textbf{Spikes}: Spikes are generally positioned in pits and kill anyone who falls on them. However it is possible to place objects over them in order to jump over these spikes.
\end{itemize}

\section*{Machineries}
The principal mechanisms in a level are:
\begin{itemize}
	\item \textbf{Buttons}: They can be pressed once to activate something, or they need a constant pressure to stay activated. The buttons can be activated by anything tangible, like characters or objects.
	\item \textbf{Blocks}: they can be pushed around the level, they can be climbed and they have a weight of their own.
	\item \textbf{Moving Platform or Elevators}: they can carry everything that is on them. Some of them are already activated, while others need to be activated with a button or other mechanisms and they can have weight restrictions.
	\item \textbf{Barriers}: Barriers are used to make inaccessible some areas. By activating a button (or something else) these barriers can be deactivated.
	\item \textbf{Terminal}: The terminals are similar to the buttons, with the difference that they can be activated and deactivated. Using them, for example, it is possible to change the direction of a laser.
	\item \textbf{Wires/Links}: The wires can be used by Bitty and allow him to cross them in order to be transported to one of the ends.
\end{itemize}

\section*{Control System}
Since some puzzles are physics-based, the movement is also based on it. The control system also takes care of the changing of the controlled character ensuring that at any time the player can use only one character at a time. To make the player understand which character he is controlling in a time, the system activates an UI indicator. The player can also control any imprisoned program, even if he will be allowed to make only small movements.



\chapter{Story and World}
The entire game is set inside a PC of a young boy, Tommy. Tommy’s PC got infected by a terrible virus which has made the pc unusable and he doesn’t know how to fix this problem. Tommy is unaware of the fact that a single brave bit is going to help him, so he is thinking to erase the disk. The protagonist has little time to destroy the virus but alone he is not enough to stop such a catastrophe, so he needs to find other programs to help him in saving his world. However, the other programs are captured by the virus and the bit needs to find a way to free them.\\
Inside the PC the world is a pyramidal structure like Dante’s Inferno (aggiungere immagine adatta), starting from the high level applications, going in the operative system level and reaching the kernel level. These macro levels are subdivided in single levels containing a puzzle.

\chapter{Art Overview}
Graphically, the levels should be invoke the idea of being inside the virtual world of a pc, having texture of string of bits and tron-like features like neon colors and lines around to represent the circuits of a pc.
The protagonist and his allies are drawn as rectangles, each one having features typical of the program they represent in their textures. They have a cartoonish style with cute faces and rounded edges. The enemies are also drawn with a cartoonish style, but they have a menacing feeling when you look at them by having angry faces and other features like edgy edges.

\chapter{Sound Overview}
The background music should be taken from the trance genre (or something similar), avoiding too much powerful tracks but looking for some more relaxing ones. This genre should be perfect to make the player feeling to be in a virtual world and to improve his focus on solving the puzzles.
The sound effects need to be thought yet.


\end{document}
